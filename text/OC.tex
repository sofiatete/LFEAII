
% RUN:
% pdflatex -output-directory=/Users/salvatorpes/Desktop/LFEAII/text/merdas /Users/salvatorpes/Desktop/LFEAII/text/OC.tex

\documentclass{article}

\author{Pedro Miguel Pombeiro Curvo (ist1102716) \\ Salvador Baptista Torpes (ist1102474) \\ Sofia Tété Garcia Ramos Nunes (ist1102633)\\ Estêvão Moreira Gomes (ist1102650)}

\usepackage[utf8]{inputenc}
\usepackage[portuguese]{babel}
\usepackage[letterpaper,top=10mm,bottom=15mm,left=10mm,right=10mm,marginparwidth=1.75cm]{geometry}
\usepackage{multicol}
\usepackage{biblatex}
\addbibresource{Bibliografia.bib}
\usepackage{graphicx}
\usepackage{subcaption}
\usepackage{tabularx}
\usepackage{booktabs}
\usepackage{array}
\usepackage{makecell}
\usepackage{multirow}
\usepackage{amsmath}
\usepackage{makecell}
\usepackage{url}
\usepackage{csquotes}
\usepackage{caption}
\usepackage{enumitem}
\usepackage{textcomp}
\usepackage{pdflscape}
\usepackage{makeidx}
% \usepackage{tocbibind}
\providecommand{\tightlist}{\relax}
\usepackage{tocloft}
\renewcommand{\cftsecindent}{0em}
\renewcommand{\cftsubsecindent}{1em}
\renewcommand{\cftsecfont}{\bfseries}
\renewcommand{\cftsubsecfont}{\itshape}
\setlength{\cftsubsecnumwidth}{0em}

\usepackage[version=4]{mhchem}
\usepackage{hyperref} % Remove "pdftex" option here
\usepackage{float}
\usepackage{fancyhdr}
\usepackage{ragged2e}
\usepackage{xkeyval}
%\usepackage{minted}
%\usemintedstyle{manni}
\usepackage{listings}
\usepackage{amssymb}



\usepackage{xcolor}
\usepackage{tikz}
\usetikzlibrary{positioning}
\usetikzlibrary{positioning, arrows.meta}
\usepackage{adjustbox}
\usepackage{sidecap}
\usepackage{graphicx}

\usepackage{tikz-3dplot}
% \usepackage{pgfplots}
\usetikzlibrary{calc, 3d, arrows}



\usetikzlibrary{shapes.geometric, arrows}


\lstset{
    language=Python,
    basicstyle=\ttfamily,
    keywordstyle=\color{blue},
    commentstyle=\color{gray},
    stringstyle=\color{orange},
    numbers=left,
    numberstyle=\tiny,
    numbersep=5pt,
    showspaces=false,
    showstringspaces=false,
    breaklines=true,
    frame=tb,
    framexleftmargin=2em,
    xleftmargin=2em,
}


%\usepackage{fontspec}

%\setmonofont{Fira Code}

\fancyhf{}
\cfoot{\thepage}
\fancyhf{} % Clear all header and footer fields
\renewcommand{\headrulewidth}{0pt} % Remove the header rule line
\cfoot{\thepage} % Set the page number in the center of the footer

\pagestyle{fancy} % Apply the fancy page style

\setlength\columnsep{20pt}

\renewcommand{\familydefault}{\sfdefault}

\newenvironment{Figure}
  {\par\medskip\noindent\minipage{\linewidth}}
  {\endminipage\par\medskip}

\makeatletter
\newenvironment{figurehere}
{\def\@captype{figure}}
{}
\makeatother

\hypersetup{
  colorlinks,
  linkcolor=blue,
  anchorcolor=black,
  citecolor=cyan,
  filecolor=cyan,
  menucolor=cyan,
  urlcolor=cyan,
  bookmarksopen=true,
  bookmarksnumbered=true
}

\makeindex


\title{\vspace{-13mm}\includegraphics[width=15mm,scale=3]{images/IST_Logo.png}\\ \vspace{5mm}
LFEAII - Ótica Coerente \vspace{-5mm}}
\date{23/24}

\usepackage{sansmathfonts}
\usepackage[T1]{fontenc}
\usepackage[OT1]{fontenc}

\begin{document}

\renewcommand{\arraystretch}{1.5}
\setlength{\columnseprule}{0.4pt}
\tdplotsetmaincoords{70}{110} % Set the viewing angle
\newcolumntype{M}[1]{>{\centering\arraybackslash\vspace{#1}}m{0.5\linewidth}<{\vspace{#1}}}
\newcolumntype{C}[2]{>{\centering\arraybackslash\vspace{#1}\rule{0pt}{#1}\hspace{0pt}}m{#2}}
\newcolumntype{w}[1]{>{\centering\arraybackslash}m{#1}}

\renewcommand*\familydefault{\sfdefault} %% Only if the base font of the document is to be sans serif

\maketitle

\vspace{-5mm}

\hrulefill

\begin{multicols}{2}

\section{Teórica  - Transformada de Fourier Ótica}

\subsection{Transformada de Fourier Ótica}

A transformada de fourier ótica de um objeto é formada num dado ponto do espaço sempre que a luz proveninente do ojeto passa por uma lente: a transformada do objeto corresponde à difração que este provoca à lus quando é atravessado:
Quanto maior a frequência espacial de um determinado padrão do objeto, menor as fendas espaciais por onde a luz passa e por isso maior a difração que este provoca, ou seja, maior o ângulo de difração - assim, a luz que atravessa padrões do objeto com frequência espacial maior é difratada mais do que a luz que atravessa padrões do objeto com frequência espacial menor - assim, na imagem de fourier, quanto mais longe radialmente estiver a luz do centro, maior a frequência espacial do padrão que a originou.
\paragraph{}
A transformada de fourier é uma estrutura que se forma naturalmente sempre que um objeto não opaco se deixa atravessar por luz, provocando a difração da mesma em diferentes intensidades e ângulos consoante as diferentes frequências espaciais do objeto.
O objetivo deste trabalho é estudar os padrões da TF de diferentes objetos através do uso de um sistema ótico que, com uma lente, consegue colocar a transformada num plano onde se encontra um filtro de fourier - este plano é ampliado e fotografado para análise das transformadas de fourier dos objetos.

\subsection{Coordenadas no Espaço de Fourier}

Na montagem experimental utilizada temos duas lentes: uma lente de ampliação e uma lente de fourier: a lente de fourier é colocada depois do objeto e tem como objetivo colocar a transformada de fourier do objeto num plano perto da lente de modo a que depois o possamos observar;
No plano focal da lente de fourier, ou seja, onde se pode observar a transformada de fourier do objeto, colocamos um filtro de fourier que possui 3 círculos ajustáveis para poder tapar o ponto central mais intenso.
Por fim, seguidamente ao filtro de fourier, colocamos a segunda lente, uma lente de ampliação: esta lente é colocada de modo a que a imagem de fourier no filtro fique no plano focal da lente de ampliação. 
Colocamos ainda, depois da lente de ampliação, a câmara CCD que irá captar a imagem final - a distância entre a câmara CCD e a lente de ampliação é menor que a distância focal da mesma lente ao filtro de fourier, de modo a que a imagem final seja ampliada.
\paragraph{}
Assim, o filtro de fourier encontra-se no plano focal da lente de fourier. É no filtro de fourier que conseguimos observar a transformada de fourier do objeto formada pela lente de fourier.
A relação entre as coordenadas no espaço de fourier $(\mu, \nu)$ e as frequências espaciais do objeto que estamos a observar $(\nu_x, \nu_y)$ é dada pelas seguintes equações:

\[ \renewcommand*\familydefault{\sfdefault}
  \begin{cases}
  \mu = \lambda f \nu_x \\
  \nu = \lambda f \nu_y
\end{cases} \]

Onde $f$ é a distância focal. 
Se tivermos um objeto do qual estamos a ver a transformada de fourier então $\nu_x$ e $\nu_y$ são as, respetivamente, a frequência do objeto ao longo do eixo $x$ e do eixo $y$.
Podemos perceber que, à medida que a frequência espacial aumenta numa dada direção, a coordenada no espaço de fourier aumenta na mesma direção - pontos correspondentes a maiores frequências espaciais encontram-se mais afastados do centro do filtro de fourier.

\begin{center}
  \begin{tikzpicture}
    % Draw x-axis
    \draw[->] (-2,0) -- (2,0) node[right] {$\mu$};
    % Draw y-axis
    \draw[->] (0,-2) -- (0,2) node[above] {$\nu$};
    % Add origin point
    \fill (0,0) circle (2pt) node[above] {$O$};
    \fill (1,1) circle (2pt) node[above right] {$A$};
    \fill (-1,1) circle (2pt) node[above left] {$B$};
    \fill (-1,-1) circle (2pt) node[below left] {$C$};
    \fill (1,-1) circle (2pt) node[below right] {$D$};
    \fill (0.3,0.3) circle (2pt) node[above right] {$E$};
    \fill (-0.3,0.3) circle (2pt) node[above left] {$F$};
    \fill (-0.3,-0.3) circle (2pt) node[below left] {$G$};
    \fill (0.3,-0.3) circle (2pt) node[below right] {$H$};
    
  \end{tikzpicture}
  
    
\end{center}

Note-se que na figura acima o ponto $O$ corresponde ao centro do filtro que contém normalmente luz intensa e indesejada.
Os pontos $A$, $B$, $C$ e $D$ correspondem a frequências espaciais mais altas que as dos pontos $E$, $F$, $G$ e $H$ uma vez que estão mais afastados do centro, ou seja, têm valores de $\mu$ e $\nu$ maiores. 

\subsection{Redes de Difração}

Uma rede de difração tem múltiplas frequências espaciais próprias: todas as combinações periódicas de riscas formam uma frequência espacial diferente: a frequência própria mais alta de uma rede difração é aquelas que corresponde ao padrão com todas as riscas.
Por outro lado, a menor frequência corresponde ao padrão com apenas a primeira e a última risca: assim, a imagem de fourier de uma rede de difração é um conjunto concêntrico de pontos, sendo que à medida que nos afastamos do centro, a frequência espacial responsável pelo ponto é maior.

\subsection{Relação Matemática}

As transformadas de fourier óticas que observamos são descritas matematicamente pela transformada de fourier de um conjunto de funções 2D. 
Se a nossa imagem for descrita pelo gráfico de uma função $f(x,y)$, então a sua transformada de fourier é dada por:
\[ \mathcal{F}(f(x,y))(\mu, \nu) = \int_{-\infty}^{\infty} \int_{-\infty}^{\infty} f(x,y) e^{-2\pi i (\mu x + \nu y)} dx dy \]
O imagem de fourier é então o gráfico da função $\mathcal{F}(f(x,y))(\mu, \nu)$ no plano $(\mu, \nu)$.
\paragraph{}
Em adição, como já vimos anteriormente, a relação entre as coordenadas no espaço de fourier $(\mu, \nu)$ e as frequências espaciais de um padrão do objeto na direção $x$ e $y$ $(\nu_x, \nu_y)$ é dada por:
\[ \begin{cases}
  \mu = \lambda f \nu_x \\
  \nu = \lambda f \nu_y
\end{cases} \]

\section{Teórica - Interferometria}

Com o objetivo de estudar a interferência de ondas eletromagnéticas em fases diferentes, utilizamos um interferómetro de Michelson.
O interferómetro de Michelson é composto por uma fonte de luz, um beam splitter, dois espelhos e uma câmara CCD. 

\subsection{Funcionamento do Interferómetro de Michelson}

A luz emitida pelo laser é direcionada (possivelemnte com a ajuda de um espelho) para o beam splitter.
O beam splitter reflete o feixe em duas direções perpendiculares: uma direciona-se a um espelho fixo e é refletido de volta para o beam splitter e do beam splitter para a fonte; o outro feixe é refletido para um espelho móvel (a sua distância o BS varia) e é refletido de volta para o BS e do BS para a câmara CCD.
Em adição, o espelho móvel tem um ligeiro tilt de modo a que o feixe que é refletido para a câmara CCD tenha um ângulo de incidência diferente do feixe que é refletido para o espelho fixo: o feixe proveninente do espelho fixo incide perpendicularmente ao CCD enquanto que o feixe proveniente do outro feixe tem um ângulo de incidência igual a $\theta$.
\paragraph{} 
Assim, existem dois fenómenos que estamos a observar: 
\paragraph{Fenómenos Geométricos - Interferência}
Em primeiro lugar, se mantivermos os dois espelhos a uma mesma distância $d_0$ do beam splitter e fizermos varia o ângulo de incidência de um dos feixes, podemos observar a formação do padrão de interferência: à medida que a diferença entre os dois ângulos de incidência aumenta, a distância entre os duas riscas no padrão de interferência torna-se cada vez menor - 
os picos de intensidade do padrão correspondem aos cruzamentos das riscas oblíquas (onda com inclinação $\theta$) com as riscas verticais (onda com inclinação $0$).
A relação entre a distância $a$ entre as riscas de interferência no padrão e o ângulo $\theta$ é dada por:
\[ a = \frac{\lambda}{\sin \theta} \]
Onde $\lambda$ é o comprimento de onda da luz do laser.
\paragraph{Fenómenos Temporais - Contraste}
Em segundo lugar, depois de fixarmos um ângulo $\theta$ para o qual o padrão de interferência seja razoável (uma distância entre discas não demasiado elevada), podemos fazer variar a distância do espelho inclinado ao beam splitter: mantemos um dos espelhos a uma distância $d_0$ e colocamos uma outra a uma distância $d_0 + \Delta d$ - ao fazer isto, vamos criar um desfasamento temporal entre a chegada dos dois feixes à CCD dado que, depois do feixe inicial incidir no beam splitter e se dividir, um deles percorre uma distância maior:
O desfasamento temporal entre a chegada dos dois feixes à CCD é dado por:
\[ \Delta t = \frac{2 \Delta d}{c} \]
\[ \Delta d = d_2 - d_1 \]
Onde assumimos que a velocidade da luz é igual à do vácuo e $d_2$ é a distância ao espelho móvel e $d_1$ é a distância ao espelho fixo.
\paragraph{}
Á medida que aumenta o desfasamento temporal entre os dois feixes, dado que estes são pulsos sinusoidais e não senos infinitos, os feixes deixam de se sobrepor e passamos a ter menos constraste - muito ruido e pouca interferência.
O contraste máximo que podemos observar acontece quando o desfasamento temporal é nulo e os dois feixes estão em fase.
\paragraph{}
Em adição, o contraste pode ser calculado pela seguinte equação:
\[ K = \frac{I_{max} - I_{min}}{I_{max} + I_{min}} \]
Onde $I_{max}$ é a intensidade máxima e $I_{min}$ é a intensidade mínima do padrão de interferência.
O objetivo nesta segunda parte é poder desenhar um gráfico do contraste da imagem de interferência obtida (para um $\theta$ constante) em função do desfasamento temporal $\Delta t$ entre os dois feixes.
A relação entre o contraste como função do desfasamento $K(\Delta t)$ e o espetro de energia $I(\omega)$ é dada pela transformada de fourier:
\[ K(\Delta t) = \mathcal{F}(I(\omega))(\Delta t) \]
Onde $\omega$ é a frequência angular da onda e $I(\omega)$ é o espetro de energia da onda.
Quando fazemos a transformada de fourier de $I$ obtemos o contraste do padrão de interferência como função do desfasamento temporal.

\subsection{Equação da Difração}

A equação da difração permite-nos relacionar o comprimento de onda da onda com o ângulo entre os dois feixes que estão a sofrer interferência. 

\subsection{Cálculo do ângulo $\theta$}

O ângulo $\theta$ calcula-se com a íris totalmente fechada uma vez que assim podemos tirar uma foto onde sejam visíveis/separáveis os dois feixes de luz que estão a sofrer interferência: como o feixe fica muito muito fino podemos vê-los distintos e calcular a distância entre eles através do comprimento dos pixeis do CCD. 
Assim sendo, a distância entre os dois feixes $\Delta_{CCD}$ corresponde ao cateto oposto de um triângulo de cujo cateto adjacente é a distância $L$ entre o CCD e o espelho móvel onde um dos feixes está a ser refletido com inclinação. Assim, o ângulo $\theta$ é dado por:
\[ \theta = \arctan \left( \frac{\Delta_{CCD}}{L} \right) \Leftrightarrow \tan \theta = \frac{\Delta_{CCD}}{L} \]

\subsection{Amplificação da Imagem}

Entre o BS e o CCD colocamos uma lente de ampliação de modo a que a imagem de interferência seja ampliada e possamos observar melhor o padrão de interferência.
De modo a sabermos qual a distância a que deviamos colocar a lente para obter a ampliação pretendida, usamos a fórmula da lente:

\[ \frac{1}{f} = \frac{1}{d_o} + \frac{1}{d_i} \]

Onde $f$ é a distância focal da lente, $d_o$ é a distância entre a lente e o objeto (neste caso, o BS) e $d_i$ é a distância entre a lente e a imagem, neste caso, o CCD.
\paragraph{}
A magnificação $M$ da imagem é dada por:

\[ M = - \frac{d_i}{d_o} \]

\subsection{Desvio de fase no Interferograma}

Como é que podemos calcular o desvio de fase entre o feixe emitido pelo laser e o feixe resultante da interferência?

\subsection{Variação com a temperatura}

De modo a averiguar como varia o padrão de interferência com a alteração da temperatura do meio, colocamos um ferro de soldar entre o BS e o espelho fixo.
À medida que a temperatura aumentava, observamos que as riscas do padrão de interferênca ficavam desformadas à medida que a temperatura aumentava.

\subsection{Variação com a distância do 2º espelho}

Fizemos variar a distância do espelho móvel ao BS e observamos que as riscas do padrão de interferência ficavam mais juntas à medida que a distância aumentava - riscas mais finas - menos $a$.

\subsection{Variação com Ondas de Pressão}

Com o objetivo de estudar como varia o padrão de interferência com ondas sonoras de determinada frequência. 
Colocamos uma coluna num suporte externo à mesa ótica para evitar que as vibrações mecânicas se propaguem para os espelhos do interferómetro. 
Verificamos que as ondas sonoras provocam um movimento do padrão de riscas de interferência: o padrão oscila como um todo a diferentes frequências consoante a frequência do som que é emitido.

\subsection{Variação com o ângulo de incidência}

Pretendemos agora estudar como varia o padrão de interferência com o ângulo de tilt do espelho móvel.
Tiramos várias fotos do padrão à medida que variamos o ângulo.

\section{Análise de Dados}

\subsection{Tratamento de imagem}

Existem dois tipos de imagens com as quais podemos trabalhar:
Em primeiro lugar temos as imagens RGB - estas são imagens cuja informação está guardada numa matriz onde o número de linhas e colunas é o número de pixeis na vertical e na horizontal da imagem. 
Cada entrada da matriz é um vetor com 3 entradas: $[R, G, B]$ onde $R$, $G$ e $B$ são as intensidades de vermelho, verde e azul respetivamente:

\[ \text{image} = \begin{bmatrix}
  [R, G, B] & [R, G, B] & \cdots & [R, G, B] \\
  [R, G, B] & [R, G, B] & \cdots & [R, G, B] \\
  \vdots & \vdots & \ddots & \vdots \\
  [R, G, B] & [R, G, B] & \cdots & [R, G, B] \\ 
\end{bmatrix} \]

\[ \qquad \text{onde} \qquad \begin{cases}
  R \in [0, 255] \\
  G \in [0, 255] \\
  B \in [0, 255]
\end{cases} \]

Em segundo lugar, temos as imagens \textbf{greyscale} - estas imagens são semelhantes mas a informação associada a cada ponto não é um array com três números mas sim um número entre 0 e 1 que representa a intensidade de cinzento do pixel:

\[ \text{image} = \begin{bmatrix}
  I & I & \cdots & I \\
  I & I & \cdots & I \\
  \vdots & \vdots & \ddots & \vdots \\
  I & I & \cdots & I \\
\end{bmatrix} \]

\subsection{Ótica de Fourier}

\subsubsection{Calibração}

Para a calibração, utilizamos as seguintes imagens: em cada uma foram feitas duas linhas paralelas aos limites da craveira e duas linhas perpendiculares às anteriores. Foi medida a distância, em pixeis, entre os pontos de interseção das linhas paralelas e perpendiculares - essa foi a distância utilizada para a calibração.
A incerteza de cada pixel é de 1 pixel uma vez que a a imagem 2D funciona como um histograma onde cada pixel é um bin - apenas podemos estar num pixel ou noutro, não há valores intermédios.
A distância entre dois pontos $(x_1, y_1)$ e $(x_2, y_2)$ em pixeis é dada por:

\[ d = \sqrt{(x_1 - x_2)^2 + (y_1 - y_2)^2} \]

Sabendo que os erros de $x_1$, $x_2$, $y_1$ e $y_2$ são todos iguais a $\pm 1$ pixel, a incerteza da distância $d$ é dada por:

\begin{align*}
  \delta d &= \left| \frac{x_1 - x_2}{\sqrt{(x_1 - x_2)^2 + (y_1 - y_2)^2}} \right| + \left| \frac{x_2 - x_1}{\sqrt{(x_1 - x_2)^2 + (y_1 - y_2)^2}} \right| + \\
  &+ \left| \frac{y_1 - y_2}{\sqrt{(x_1 - x_2)^2 + (y_1 - y_2)^2}} \right| + \left| \frac{y_2 - y_1}{\sqrt{(x_1 - x_2)^2 + (y_1 - y_2)^2}} \right|
\end{align*}


\begin{figure}[H]
  \centering
  \includegraphics[width=0.9\linewidth]{../graphs/craveira2mm_AB_5aula_calibration.png}
  \caption{Calibração da craveira 2mm}
\end{figure}

\begin{figure}[H]
  \centering
  \includegraphics[width=0.9\linewidth]{../graphs/craveira3mm_AB_5aula_calibration.png}
  \caption{Calibração da craveira 3mm}
\end{figure}

\begin{figure}[H]
  \centering
  \includegraphics[width=0.9\linewidth]{../graphs/craveira4mm_AB_5aula_calibration.png}
  \caption{Calibração da craveira 4mm}
\end{figure}

\begin{figure}[H]
  \centering
  \includegraphics[width=0.9\linewidth]{../graphs/craveira4mm2_AB_5aula_calibration.png}
  \caption{Calibração da craveira 4mm 2}
\end{figure}

Em seguida, realizou-se uma regressão linear entre os valores em metros e em pixeis para obter uma calibração:

\begin{figure}[H]
  \centering
  \includegraphics[width=0.9\linewidth]{../graphs/regr_calibration.png}
  \caption{Calibração}
\end{figure}

A reta de ajuste foi:

\[ \text{Distance (m)} = 8.144 \times 10^{-6} \times \text{Pixels} + 9.141 \times 10^{-5} \]

\subsubsection{Filtro AB}

O nosso objetivo é agora determinar as frequências espaciais do filtro AB utilizado



\end{multicols}





\newpage

\section{Montagem - Interferómetro de Michelson}

\vspace{3cm}

\begin{center}
\begin{tikzpicture}
  % Beam splitter (BS)
  \draw (-1,-1) rectangle (1,1);
  \node at (1.3,1.3) {BS};
  
  % Labels
  % \node at (0,0) {O};

  % Incident light beam
  \draw[->] (-8,0) -- (-1,0);
  \node at (-4.5,0.3) {Light ($\lambda$)};
  \draw[dotted, -] (-1,0) -- (0,0);
  \draw[dotted, ->] (0,0) -- (1,0);
  \draw[dotted, ->] (0,0) -- (0,1);
  \draw[dotted, ->] (0,0) -- (0,-1);
  \draw[<->] (1,0) -- (5,0);
  \node at (3,0.3) {$d_0$};
  \draw[<->] (0,1) -- (0,5);
  \node at (0.3,3) {$d_0$};
  \draw[->] (0,-1) -- (0,-6);
  % \draw[-] (0.5,-1.5) -- (1.5, -0.5);
  \draw[-, blue] (-1,-1) -- (1, 1);


  \draw (-1,5) rectangle (1,5.2);
  \fill[yellow!50] (-1,5) rectangle (1,5.2);
  \draw[rotate=-6, dashed, red] (-1.5,5) rectangle (0.5,5.2);

  \draw[dashed] (-1,7) rectangle (1,7.2);
  \fill[yellow!50] (-1,7) rectangle (1,7.2);
  \draw[rotate=-6, dashed, red] (-1.7,7) rectangle (0.3,7.2);


  \draw[->] (0,5.2) -- (0,7);
  \node at (0.3,6.1) {$\Delta d$};

  \draw (5,-1) rectangle (5.2,1);
  \fill[yellow!50] (5,-1) rectangle (5.2,1);

  % Deviated ray of light:
  \draw[dashed, ->, red] (0,5) -- (-0.7,-6);
  \draw[dashed, ->, red] (0,7) -- (-0.9,-6);

  % Laser:
  \draw (-8.2,-1) rectangle (-8,1);
  \node at (-8.2,1.4) {Laser};

  % Angle:
  \draw (-0.6,-4) arc (231:270:1);
  \node at (-0.3,-4.5) {$\theta$};

  % CCD:

  \draw (-1,-6) rectangle (1,-6.6);
  \node at (0,-6.3) {CCD};

  % Image from CCD:
  \draw[dashed] (1,-6) -- (3,-4.6);
  \draw[dashed] (1,-6.6) -- (3,-8);
  \draw (3,-8) rectangle (7,-4.6);
  % Draw lines of the diffraction pattern
  \draw[-] (3.1,-6.6) -- (6.9,-6.6);
  \draw[-] (3.1,-6) -- (6.9,-6);
  \draw[-] (3.1,-5.4) -- (6.9,-5.4);
  \draw[-] (3.1,-4.8) -- (6.9,-4.8);
  \draw[-] (3.1,-7.2) -- (6.9,-7.2);
  \draw[-] (3.1,-7.8) -- (6.9,-7.8);
  \draw[<->] (4,-6.6) -- (4,-6);
  \node at (5.3,-6.3) {$a=\lambda / \sin(\theta)$};
  \node at (5.1,-4.1) {Pattern of Interference};

  
  
  
\end{tikzpicture}
\end{center}

\newpage 

\section*{LightPipes Simulation}

% \begin{multicols}{2}

\begin{enumerate}
  \item Criar o plano da simulação grid - tenho de dizer o tamanho total GridSize(), lambda, GridDimension() (=1024)
  \item criar o feixe - begin()
  \item criar o objeto - aperture() - objeto opaco ou screen() - objeto normal
  \item propagar o feixe até à lente - exsitem várias - Forvard() ou Fresnel(): Field3 = Forvard(distância  a percorrer, Field2)
  \item propagar a simulação a distância entre o objeto e a lente;
  \item Simular a lente - lens() - é uma função de progação através da lente;
  \item propagar a simulação a distância entre a lente e o plano de fourier - a distância focal da lente;
  \item Se quiser bloquear altas ou baixas frequências, clocamos respetivamente uma circular aperture e um circular screen;
  \item Informação guardada no campo: intensidade, fase;
  \item Objeto branco no fundo preto em png: aperture;
  \item Objeto preto no fundo branco em png: screen;
\end{enumerate}

% \end{multicols}

\newpage

\section*{Aula Apresentação LFEUI}

\begin{enumerate}
  \item Até 27 de outubro - escolher uma experiência;
  \item 3 Novembro - enviar lista com os pares grupos-experiência;
  \item 6 Novembro - Leilão dos grupos em falta;
  \item 10 Novmebro - Verificação junto dos Responsáveis;
\end{enumerate}

\end{document}