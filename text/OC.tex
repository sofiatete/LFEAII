
% RUN:
% pdflatex -output-directory=/home/salvadortorpes/LFEAII/text/merdas /home/salvadortorpes/LFEAII/text/OC.tex

\documentclass{article}

\author{Pedro Miguel Pombeiro Curvo (ist1102716) \\ Salvador Baptista Torpes (ist1102474) \\ Sofia Tété Garcia Ramos Nunes (ist1102633)\\ Estêvão Moreira Gomes (ist1102650)}

\usepackage[utf8]{inputenc}
\usepackage[portuguese]{babel}
\usepackage[letterpaper,top=10mm,bottom=15mm,left=10mm,right=10mm,marginparwidth=1.75cm]{geometry}
\usepackage{multicol}
\usepackage{biblatex}
\addbibresource{Bibliografia.bib}
\usepackage{graphicx}
\usepackage{subcaption}
\usepackage{tabularx}
\usepackage{booktabs}
\usepackage{array}
\usepackage{makecell}
\usepackage{multirow}
\usepackage{amsmath}
\usepackage{makecell}
\usepackage{url}
\usepackage{csquotes}
\usepackage{caption}
\usepackage{enumitem}
\usepackage{textcomp}
\usepackage{pdflscape}
\usepackage{makeidx}
% \usepackage{tocbibind}
\providecommand{\tightlist}{\relax}
\usepackage{tocloft}
\renewcommand{\cftsecindent}{0em}
\renewcommand{\cftsubsecindent}{1em}
\renewcommand{\cftsecfont}{\bfseries}
\renewcommand{\cftsubsecfont}{\itshape}
\setlength{\cftsubsecnumwidth}{0em}

\usepackage[version=4]{mhchem}
\usepackage{hyperref} % Remove "pdftex" option here
\usepackage{float}
\usepackage{fancyhdr}
\usepackage{ragged2e}
\usepackage{xkeyval}
%\usepackage{minted}
%\usemintedstyle{manni}
\usepackage{listings}
\usepackage{amssymb}



\usepackage{xcolor}
\usepackage{tikz}
\usetikzlibrary{positioning}
\usetikzlibrary{positioning, arrows.meta}
\usepackage{adjustbox}
\usepackage{sidecap}
\usepackage{graphicx}

\usepackage{tikz-3dplot}
% \usepackage{pgfplots}
\usetikzlibrary{calc, 3d, arrows}



\usetikzlibrary{shapes.geometric, arrows}


\lstset{
    language=Python,
    basicstyle=\ttfamily,
    keywordstyle=\color{blue},
    commentstyle=\color{gray},
    stringstyle=\color{orange},
    numbers=left,
    numberstyle=\tiny,
    numbersep=5pt,
    showspaces=false,
    showstringspaces=false,
    breaklines=true,
    frame=tb,
    framexleftmargin=2em,
    xleftmargin=2em,
}


%\usepackage{fontspec}

%\setmonofont{Fira Code}

\fancyhf{}
\cfoot{\thepage}
\fancyhf{} % Clear all header and footer fields
\renewcommand{\headrulewidth}{0pt} % Remove the header rule line
\cfoot{\thepage} % Set the page number in the center of the footer

\pagestyle{fancy} % Apply the fancy page style

\setlength\columnsep{20pt}

\renewcommand{\familydefault}{\sfdefault}

\newenvironment{Figure}
  {\par\medskip\noindent\minipage{\linewidth}}
  {\endminipage\par\medskip}

\makeatletter
\newenvironment{figurehere}
{\def\@captype{figure}}
{}
\makeatother

\hypersetup{
  colorlinks,
  linkcolor=blue,
  anchorcolor=black,
  citecolor=cyan,
  filecolor=cyan,
  menucolor=cyan,
  urlcolor=cyan,
  bookmarksopen=true,
  bookmarksnumbered=true
}

\makeindex


\title{\vspace{-13mm}\includegraphics[width=15mm,scale=3]{images/IST_Logo.png}\\ \vspace{5mm}
LFEAII - Ótica Coerente \vspace{-5mm}}
\date{23/24}

\usepackage{sansmathfonts}
\usepackage[T1]{fontenc}
\usepackage[OT1]{fontenc}

\begin{document}

\renewcommand{\arraystretch}{1.5}
\setlength{\columnseprule}{0.4pt}
\tdplotsetmaincoords{70}{110} % Set the viewing angle
\newcolumntype{M}[1]{>{\centering\arraybackslash\vspace{#1}}m{0.5\linewidth}<{\vspace{#1}}}
\newcolumntype{C}[2]{>{\centering\arraybackslash\vspace{#1}\rule{0pt}{#1}\hspace{0pt}}m{#2}}
\newcolumntype{w}[1]{>{\centering\arraybackslash}m{#1}}

\renewcommand*\familydefault{\sfdefault} %% Only if the base font of the document is to be sans serif

\maketitle

\vspace{-5mm}

\hrulefill

\begin{multicols}{2}

\section{Teórica}

\subsection{TF Ótica}

A transformada de fourier ótica de um objeto é formada num dado ponto do espaço sempre que a luz proveninente do ojeto passa por uma lente: a transformada do objeto corresponde à difração que este provoca à lus quando é atravessado:
Qaundo maior a frequência espacial de um determinado padrão do objeto, menor as fendas espaciais por onde a luz passa e por isso maior a difração que este provoca, ou seja, maior o ângulo de difração - assim, a luz que atravessa padrões do objeto com frequência espacial maior é difratada mais do que a luz que atravessa padrões do objeto com frequência espacial menor - assim, na imagem de fourier, quanto mais longe radialmente estiver a luz do centro, maior a frequência espacial do padrão que a originou.

\subsection{Coordenadas no Espaço de Fourier}

Na montagem experimental utilizada temos duas lentes: uma lente de ampliação e uma lente de fourier: a lente de fourier é colocada depois do objeto e tem como objetivo colocar a transformada de fourier do objeto num plano perto da lente de modo a que depois o possamos observar;
No plano focal da lente de fourier, ou seja, onde se pode observar a transformada de fourier do objeto, colocamos um filtro de fourier que possui 3 círculos ajustáveis para poder tapar o ponto central mais intenso.
Por fim, seguidamente ao filtro de fourier, colocamos a segunda lente, uma lente de ampliação: esta lente é colocada de modo a que a imagem de fourier no filtro fique no plano focal da lente de ampliação. 
Colocamos ainda, depois da lente de ampliação, a câmara CCD que irá captar a imagem final - a distância entre a câmara CCD e a lente de ampliação é menor que a distância focal da mesma lente ao filtro de fourier, de modo a que a imagem final seja ampliada.
\paragraph{}
Assim, o filtro de fourier encontra-se no plano focal da lente de fourier. É no filtro de fourier que conseguimos observar a transformada de fourier do objeto formada pela lente de fourier.
A relação entre as coordenadas no espaço de fourier $(\mu, \nu)$ e as frequências espaciais do objeto que estamos a observar $(\nu_x, \nu_y)$ é dada pelas seguintes equações:

\[ \begin{cases}
  \mu = \lambda f \nu_x \\
  \nu = \lambda f \nu_y
\end{cases} \]

Onde $f$ é a distância focal. 
Se tivermos um objeto do qual estamos a ver a transformada de fourier então $\nu_x$ e $\nu_y$ são as, respetivamente, a frequência do objeto ao longo do eixo $x$ e do eixo $y$.
Podemos perceber que, à medida que a frequência espacial aumenta numa dada direção, a coordenada no espaço de fourier aumenta na mesma direção - pontos correspondentes a maiores frequências espaciais encontram-se mais afastados do centro do filtro de fourier.

\section{Procedimento Experimental}

\end{multicols}

\end{document}