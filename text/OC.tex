
% RUN:
% pdflatex -output-directory=/home/salvadortorpes/LFEAII/text/merdas /home/salvadortorpes/LFEAII/text/OC.tex

\documentclass{article}

\author{Pedro Miguel Pombeiro Curvo (ist1102716) \\ Salvador Baptista Torpes (ist1102474) \\ Sofia Tété Garcia Ramos Nunes (ist1102633)\\ Estêvão Moreira Gomes (ist1102650)}

\usepackage[utf8]{inputenc}
\usepackage[portuguese]{babel}
\usepackage[letterpaper,top=10mm,bottom=15mm,left=10mm,right=10mm,marginparwidth=1.75cm]{geometry}
\usepackage{multicol}
\usepackage{biblatex}
\addbibresource{Bibliografia.bib}
\usepackage{graphicx}
\usepackage{subcaption}
\usepackage{tabularx}
\usepackage{booktabs}
\usepackage{array}
\usepackage{makecell}
\usepackage{multirow}
\usepackage{amsmath}
\usepackage{makecell}
\usepackage{url}
\usepackage{csquotes}
\usepackage{caption}
\usepackage{enumitem}
\usepackage{textcomp}
\usepackage{pdflscape}
\usepackage{makeidx}
% \usepackage{tocbibind}
\providecommand{\tightlist}{\relax}
\usepackage{tocloft}
\renewcommand{\cftsecindent}{0em}
\renewcommand{\cftsubsecindent}{1em}
\renewcommand{\cftsecfont}{\bfseries}
\renewcommand{\cftsubsecfont}{\itshape}
\setlength{\cftsubsecnumwidth}{0em}

\usepackage[version=4]{mhchem}
\usepackage{hyperref} % Remove "pdftex" option here
\usepackage{float}
\usepackage{fancyhdr}
\usepackage{ragged2e}
\usepackage{xkeyval}
%\usepackage{minted}
%\usemintedstyle{manni}
\usepackage{listings}
\usepackage{amssymb}



\usepackage{xcolor}
\usepackage{tikz}
\usetikzlibrary{positioning}
\usetikzlibrary{positioning, arrows.meta}
\usepackage{adjustbox}
\usepackage{sidecap}
\usepackage{graphicx}

\usepackage{tikz-3dplot}
% \usepackage{pgfplots}
\usetikzlibrary{calc, 3d, arrows}



\usetikzlibrary{shapes.geometric, arrows}


\lstset{
    language=Python,
    basicstyle=\ttfamily,
    keywordstyle=\color{blue},
    commentstyle=\color{gray},
    stringstyle=\color{orange},
    numbers=left,
    numberstyle=\tiny,
    numbersep=5pt,
    showspaces=false,
    showstringspaces=false,
    breaklines=true,
    frame=tb,
    framexleftmargin=2em,
    xleftmargin=2em,
}


%\usepackage{fontspec}

%\setmonofont{Fira Code}

\fancyhf{}
\cfoot{\thepage}
\fancyhf{} % Clear all header and footer fields
\renewcommand{\headrulewidth}{0pt} % Remove the header rule line
\cfoot{\thepage} % Set the page number in the center of the footer

\pagestyle{fancy} % Apply the fancy page style

\setlength\columnsep{20pt}

\renewcommand{\familydefault}{\sfdefault}

\newenvironment{Figure}
  {\par\medskip\noindent\minipage{\linewidth}}
  {\endminipage\par\medskip}

\makeatletter
\newenvironment{figurehere}
{\def\@captype{figure}}
{}
\makeatother

\hypersetup{
  colorlinks,
  linkcolor=blue,
  anchorcolor=black,
  citecolor=cyan,
  filecolor=cyan,
  menucolor=cyan,
  urlcolor=cyan,
  bookmarksopen=true,
  bookmarksnumbered=true
}

\makeindex


\title{\vspace{-13mm}\includegraphics[width=15mm,scale=3]{images/IST_Logo.png}\\ \vspace{5mm}
LFEAII - Ótica Coerente \vspace{-5mm}}
\date{23/24}

\usepackage{sansmathfonts}
\usepackage[T1]{fontenc}
\usepackage[OT1]{fontenc}

\begin{document}

\renewcommand{\arraystretch}{1.5}
\setlength{\columnseprule}{0.4pt}
\tdplotsetmaincoords{70}{110} % Set the viewing angle
\newcolumntype{M}[1]{>{\centering\arraybackslash\vspace{#1}}m{0.5\linewidth}<{\vspace{#1}}}
\newcolumntype{C}[2]{>{\centering\arraybackslash\vspace{#1}\rule{0pt}{#1}\hspace{0pt}}m{#2}}
\newcolumntype{w}[1]{>{\centering\arraybackslash}m{#1}}

\renewcommand*\familydefault{\sfdefault} %% Only if the base font of the document is to be sans serif

\maketitle

\vspace{-5mm}

\hrulefill

\begin{multicols}{2}

\section{Objetivos}

\begin{enumerate}
    \item 1º LAB: alinhamento dos lasers; obtenção de uma imagem de fourier calibrada do slide AB;
    \item 2º LAB: Obtenção das imagens de Fourier calibradas dos restantes objetos difrativos (team GRIDS, rede Ronchi,...)
\end{enumerate}

\section{Perguntas}

\paragraph{Imagem de Fourier} Uma imagem de Fourier é 
\paragraph{Fórmula de Difração de Fraunhofer} 
\[ \frac{I}{I_0} = \left( \frac{\sin \alpha}{\alpha} \right)^2 \left( \frac{\sin \beta}{\beta} \right)^2 \]
Onde $\alpha$ e $\beta$ são os ângulos de difração, e $I_0$ é a intensidade da luz incidente.
Esta fórmula é válida para padrões de difração quando a imagem de Fourier é obtida longe do objeto difrativo, ou seja, quando a distância entre o objeto difrativo e a imagem de Fourier é muito maior que a dimensão do objeto difrativo.

\paragraph{Fórmula de Difração de Fresnel}
\[ \frac{I}{I_0} = \left( \frac{J_1(\alpha)}{\alpha} \right)^2 \left( \frac{J_1(\beta)}{\beta} \right)^2 \]
Onde $\alpha$ e $\beta$ são os ângulos de difração, e $I_0$ é a intensidade da luz incidente.
Esta fórmula é válida para padrões de difração quando a imagem de Fourier é obtida perto do objeto difrativo, ou seja, quando a distância entre o objeto difrativo e a imagem de Fourier é da ordem da dimensão do objeto difrativo.

\end{multicols}

\end{document}